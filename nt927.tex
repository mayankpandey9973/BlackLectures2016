\documentclass{article}
\usepackage{amsmath, amssymb, amsthm, fullpage}
\theoremstyle{plain} 
\newtheorem{theorem}{Theorem}
\newtheorem{corollary}{Corollary}
\newtheorem{proposition}{Proposition}
\newtheorem{lemma}[theorem]{Lemma}
\numberwithin{equation}{section}

\begin{document}
\title{Number theory}
\author{RMS Black Team}
\date{}
\maketitle

\section*{Euclidean algorithm, B\'ezout, unique factorization}

A useful way of computing the greatest common divisors of integers is by using the following fact: $(a, b) = (a, a - b)$ for all 
$a, b\in\mathbb{Z}$. 
The proof that this terminates relies on the following fact: for all $a, b\in\mathbb{Z}$, there exist $q,r\in\mathbb{Z}$ s.t. 
$|r|\le \frac{|b|}{2}$. Then, we can repeatedly use the fact that $(a, b) = (r, b)$ to quickly compute the gcd of two numbers.
This process is called the Euclidean algorithm, and it is quite useful in the settings in which one can get it to work. 

\begin{lemma}[B\'ezout]
For two relatively prime $a,b$, there exist $x,y\in\mathbb{Z}$ s.t. $ax + by = 1$. 
\end{lemma}

\begin{proof}
Let $(a_1, \dots, a_n)_0 = \{a_1x_1 + \dots + a_nx_n: x_1, \dots, x_n\in\mathbb{Z}\}.$ Then, it is necessary and sufficient to show
that $(a, b)_0 = \mathbb{Z}.$ However, note that we have that $(a, b)_0 = (a, a - b)_0.$ This is highly suggestive of the Euclidean
algorithm, and in fact, by repeatedly applying this in the same way one would use the Euclidean algorithm to compute $(a, b)$, one
can show that $(a, b)_0 = (1)_0 = \mathbb{Z}$ as desired.
\end{proof}

We shall now use this to prove the following very useful fact about primes

\begin{lemma}[Euclid]
If $p$ is prime, $p|ab$, then we have that either $p|a$ or $p|b$.
\end{lemma}
\begin{proof}
If $p|a$, we are done, so suppose that $(a, p) = 1$. Otherwise, by B\'ezout, there exist $x, y\in\mathbb{Z}$ such that 
$px + ay = 1$. It follows that $pbx + aby = b$. However, note that $p|ab$, so it follows that $p|pbx + aby = b$, and the desired
result follows.
\end{proof}

With a bit of work, it is possible to use this to show the following:

\begin{theorem}[Fundamental Theorem of Arithmetic]
For all nonzero $n$ not equal to $1$, there exist $p_1, \dots, p_n$ that are unique up to permutation such that $n = \pm p_1\dots p_n$
\end{theorem}

\begin{proof}
First, we shall show that $n$ can be written as the product of primes, and then we shall show the uniqueness. Also, we shall suppose
that $n$ is positive for notational convenience.

We shall prove the first part by contradiction. 
Suppose, for the sake of contradiction, that $n$ is the smallest positive integer that cannot be written as the product
of primes. Then, clearly, $n$ is not a prime, so there exist $a, b > 1$ s.t. $n = ab$. Then, clearly, $a,b < n$. However since 
$n$ was assumed to be the smallest positive integer that could not be written as the product of primes, $a, b$ must be expressible 
as the product of primes, which is a contradiction. The desired result follows.

For showing the uniqueness, suppose that there exist $p_1, \dots, p_k$ and $\ell_1, \dots, \ell_m$ that are distinct s.t. 
$n = p_1\dots p_k = \ell_1\dots\ell_m.$ Also, take $n$ minimal, so 
that we have that $\{p_1, \dots, p_k\}$ and $\{\ell_1, \dots, \ell_m\}$ 
are disjoint (otherwise, we could divide out the primes in common and get a smaller value that breaks unique factorization). 
However, we clearly have that $p_1 | \ell_1\dots \ell_m$, so by Euclids lemma, there exists $j$ s.t. $p_1|\ell_j.$ 
However, $\ell_j$ is prime, so $p_1 = \ell_j$, which is a contradiction, as $\{p_1, \dots, p_k\}$ and $\{\ell_1, \dots, \ell_m\}$
are disjoint. The desired result follows.
\end{proof}

The fact that unique factorization is true is not always true in settings other than $\mathbb{Z}$. For example, we have that in 
$\mathbb{Z}[\sqrt{-5}]$, $(1 + \sqrt{-5})(1 - \sqrt{-5}) = 2\cdot 3$. We won't go over what exactly it means to be
prime in $\mathbb{Z}[\sqrt{-5}]$, but just be aware that unique factorization is not very obvious.

\subsection*{Problems}

\begin{enumerate}
\item Show that if $n|ab$, and $(a, b) = 1$, either $n|a$ or $n|b$.
\item Find $x, y$ such that $124x + 263y = 1$.
\item (IMO) Show that $21x + 4$ and $14x + 3$ are relatively prime for all $x\in\mathbb{Z}.$
\end{enumerate}

\section*{$\mathbb{Z}/(n)$, Chinese remainder theorem, Euler's theorem}

We say that $a$ is congruent to $b$ modulo $n$ or $a\equiv b\pmod{n}$ if $n|a - b$. It is not hard to show that if 
$a\equiv b\pmod{n}, c\equiv d\pmod{n}$, then $a + c\equiv b + d\pmod{n}, ac\equiv bd\pmod{n}.$ For all $n$, we can split 
up the integers into $n$ equivalence classes where two integers are in the same equivalence class if they are congruent
modulo $n$ (since congruence modulo $n$ is an equivalence relation). This is written as $\mathbb{Z}/(n)$ or $\mathbb{Z}/n\mathbb{Z}$.
Note that the parentheses around the $n$ are actually important in this case. For example, $\mathbb{Z}/(3)$ consists of 
the elements $\{\dots, -3, 0, 3, \dots\}, \{\dots, -2, 1, 4, \dots\}, \{\dots, -1, 2, 5, \dots\}$. We can add elements of 
$\mathbb{Z}/(n)$ by taking the sum of every pair of elements in each of the two equivalence classes we are adding. Occasionally, we 
may do things like add elements of $\mathbb{Z}$ and $\mathbb{Z}/(n)$, which isn't technically correct as we should be adding the 
equivalence class of the element of $\mathbb{Z}$. However, it will typically will be clear what I mean.

\begin{theorem}[Chinese remainder theorem]
There is a bijection $$f:\mathbb{Z}/(n_1\dots n_m)\rightarrow \mathbb{Z}/(n_1)\times\dots\times\mathbb{Z}/(n_m)$$ for pairwise 
relatively prime $n_1, \dots, n_m$. Also, we have that $f(a + b) = f(a) + f(b)$, and $f(ab) = f(a)f(b)$, where addition and
multiplication of elements in $\mathbb{Z}/(n_1)\times\dots\times\mathbb{Z}/(n_m)$ is done element-wise. 
(The Cartesian product $A\times B$ for two sets $A, B$ is the set of ordered pairs $(a, b)$ for $a\in A, b\in B$).

Equivalently, if $n\equiv k_i\pmod{n_i}$ for $1 \le i\le m$, then there exists some unique $K$ modulo $n_1\dots n_m$ s.t. 
$n\equiv K\pmod{n_1\dots n_m}$.
\end{theorem}

\begin{proof}[Sketch of proof]
We shall work with the second version, which clearly implies the first statement. It is sufficient to show that this holds for 
$m = 2$. To show the existence of solutions, one can use B\'ezout's lemma. For uniqueness, just note that if you know some integer 
modulo $n$ for some $n$, then it is determined modulo all of its factors.
\end{proof}

The use of the Chinese remainder theorem and related ideas is that one can often simply reduce a problem down to showing that 
some statement holds for prime powers.


From B\'ezout, it is also easy to see that the following holds:
\begin{lemma}[Inverses modulo $n$]
For all $a$ relatively prime to $n$, there exists some $b\in\mathbb{Z}/(n)$ s.t. $ab\equiv 1\pmod{n}$. This is called the inverse 
of $a$ modulo $n$, and is often denoted $a^{-1}$.
\end{lemma}

Now, let $(\mathbb{Z}/(n))^*$ be the set of invertible elements in $\mathbb{Z}/(n)$. Note that $(\mathbb{Z}/(n))^*$ is closed under
multiplication. Also, we define the Euler totient function 
$\varphi(n)$ to be $|(\mathbb{Z}/(n))^*|$. By the Chinese remainder theorem, $\varphi$ is multiplicative; we have that for 
relatively prime $m, n$, $\varphi(mn) = \varphi(m)\varphi(n).$ Therefore, if $n = p_1^{\alpha_1}\dots p_k^{\alpha_k}$ for primes 
$p_1,\dots, p_k$, $\varphi(n) = (p_1^{\alpha_1} - p_1^{\alpha_1 - 1})\dots (p_k^{\alpha_k} - p_k^{\alpha_k - 1}).$

Also, we have the following:

\begin{theorem}[Euler's totient theorem]
For all $(a, n) = 1$, we have $a^{\varphi(n)} \equiv 1\pmod{n}$
\end{theorem}

\begin{proof} 
It is not hard to show that multiplication by some element $a\in(\mathbb{Z}/(n))^*$ simply permutes the elements of 
$(\mathbb{Z}/(n))^*$. Now, let $$N = \prod_{k\in(\mathbb{Z}/(n))^*} k.$$ Then, we also have that 
$$N = \prod_{k\in(\mathbb{Z}/(n))^*} ak = a^{\varphi(n)}N$$ for all $a \in (\mathbb{Z}/(n))^*$. It follows that since 
$N\in(\mathbb{Z}/(n))^*$, it is invertible, so we have that $a^{\varphi(n)} = 1$ in $(\mathbb{Z}/(n))^*$, as desired.
\end{proof}

\begin{corollary}[Fermat's little theorem]
For all $a, p$ s.t. $p\nmid a$, $a^{p - 1}\equiv 1\pmod{p}$.
\end{corollary}
\subsection*{Problems}

\begin{enumerate}
\item Find the smallest positive integer that is congruent to $1\pmod{2}$, $0\pmod{3}$, $2\pmod{5}$, and $10\pmod{13}$.
\item Find last 3 digits of $2^{2^{2^{2^{2^2}}}}$.
\item Find the number of $0 < a\le 1001$ such that $a, a + 1,$ and $ a + 2$ are all relatively prime to $1001$.
\item (Wilson's theorem) Show that $(p - 1)!\equiv -1\pmod{p}$ for all primes $p$.
\item Show that for odd primes $p$, $\left(\left(\frac{p - 1}{2}\right)!\right)^2\equiv 1\pmod{p}$ 
if $p\equiv 3\pmod{4}$ and $-1\pmod{p}$ otherwise.
\item A number is \textit{squarefree} if it is not divisible by the square of any prime. 
Show that there are arbitrarily large sequences of consecutive numbers that are not squarefree.
\item (2015 December TST) Prove that for every $n\in \mathbb N$, there exists a set $S$ of $n$ positive integers such that for any 
two distinct $a,b\in S$, $a-b$ divides $a$ and $b$ but none of the other elements of $S$.
\end{enumerate}

\section*{Orders and primitive roots}
More stuff to come.

\section*{Arithmetical functions}

An arithmetical function is any map $f:\mathbb N\rightarrow \mathbb{C}$. For example, $\varphi$ is an arithmetical function. 
Define the Dirichlet convolution of two arithmetical functions $f,g$ as $$f*g = \sum_{d|n} f(d)g\left(\frac{n}{d}\right).$$
Note that the Dirichlet convolution distributes over addition, and if its inputs are two multiplicative
functions, it gives a multiplicative function. The nice thing about multiplicative functions on computational problems is that
they are quite easy to compute, as one simply has to be able to evaluate the multiplicative function at prime powers. 

\begin{theorem} 

We have that the following hold: 

\begin{enumerate}
\item $f*g = g*f$
\item $(f*g)*h = f*(g*h)$
\end{enumerate}
\end{theorem}
\begin{proof}
The first one follows immediately from the fact that divisors $d$ of $n$ correspond to divisors $\frac{n}{d}$ of $n$.

The second one follows from the definition of divisor convolutions like so: 

$$h*(f*g) = (f*g)*h = \sum_{ab = n}\left(\sum_{cd = a}f(c)g(d)\right)h(b) = \sum_{bcd = n}f(c)g(d)h(b).$$

Note that this is symmetric in $f, g, h$, so the desired result follows by permuting $f, g, h$ and using commutativity.

\end{proof}

Let the identity arithmetical function $\delta(n) = 1$ if $n = 1$ and $0$ otherwise. Then, this satisfies the following very 
nice identity: $f*\delta = f$. Also, define the M\"obius function $\mu(n)$ to be $0$ if n is not squarefree, and $(-1)^{\omega(n)}$
if $n$ is squarefree, where $\omega(n)$ is the number of distinct prime divisors of $n$. Also, let $1(n) = 1$ for all $n$.

Then, we have the following:

\begin{lemma}
$$\mu * 1 = \delta$$
\end{lemma}

\begin{proof}
Note that $\mu$ is multiplicative, as is $\delta$ so it is sufficient to check that this holds for prime powers. However, this follows
trivially, as $\sum_{d|p^k}\mu(d) = \mu(p^0) +\dots + \mu(p^k)$. If $k = 0$, then we are done. Otherwise, this equals 
$\mu(1) + \mu(p) = 0 = \delta(p^k)$ as desired, since $\mu$ vanishes on powers of $p$ greater than $p$.
\end{proof}

\begin{theorem}[M\"obius inversion]
For any arithmetical functions $f,g$, we have
$$f(n) = \sum_{d|n}g(d) \iff g(n) = \sum_{d|n} f(d)\mu\left(\frac{n}{d}\right).$$

\end{theorem}

\begin{proof}
The statement is equivalent to showing that $f = g*1\iff g = f*\mu$. 
Note that we have that $f = g*1\Rightarrow f*\mu = (g * 1)*\mu = g * (1 * \mu) = g*\delta = g$. In the other direction, we have that
$g = f * \mu\Rightarrow g * 1 = f * \mu * 1 = f * (\mu * 1) = f * \delta = f$ as desired.
\end{proof}
\subsection*{Problems}


\begin{enumerate}
\item Call a finite sequence $S$ of 0s and 1s \textit{primitive} if there does not exist a shorter sequence that can be repeated to 
form $S$. For example, $110101$ is primitive, while $101101$ is not. Compute the number of primitive sequences of length $2015$.
\item Call an integer $3$-free if it is not divisible by $p^3$ for any prime $p$. Find the number of $3$-free integers less than 1000.
\item Compute $\tau(1) + \dots + \tau(100)$.

\end{enumerate}
\end{document}
